
\section{Conclusion}
In this paper, we have presented an experimental protocol for the comparison of
numerical methods on a Kelvin-Helmholtz instability using topological analysis.
An ensemble dataset of 180 members has been computed for this instability by a
simulation code developed in our institution and running on a supercomputer.
While traditional approaches based on the kinetic energy (\autoref{energie}) only enable
to validate the physical conformity of the generated flow,
% to assess the physical conformity of the generated flow,
our overall approach provides finer analyses. In particular,
% Based on the assumptions made in the literature (\autoref{sec_simulation}),
the
protocol using the persistence curves (\autoref{sec_persistence}) allowed us to
observe differences between the TENO and WENO-Z reconstructions. It also
confirms an independence of the
reconstruction order (5 or 7) when
using the TENO scheme allowing
% speedup
practical
computational speedup,
without loss of precision. The protocol
based on the Wasserstein distance (\autoref{sec_outlier}) succeeded in
discriminating the HLL solvers from other configurations, validating the use of
such a topological analysis to confirm domain field expectations. The last
protocol, based on recent clustering methods (\autoref{sec_khi_cluster})
successfully differentiates the topology of computations based on FDS (Flux
Difference Splitting) and FTS (Flux Type Splitting) solvers.
Overall, the validation of the hypotheses reported by CFD experts (\autoref{sec_hypotheses}) provides reliable indications for the tuning of a flow simulation, to help CFD users achieve the best balance between computation accuracy and speed.
%
%
% in contrast to traditional approaches (\autoref{energie}), which only validate the physical plausibility of a flow, our framework enabled the validation of the above hypotheses (formalized in \autoref{sec_hypotheses})}
% In particular, the validation of these hypotheses are a practical contribution towards }
% Flux Differe
% (FDS) and FTS solver computations.


The results obtained in this experimental study also show
the viability of topological methods for the representation and comparison of
Kelvin-Helmholtz
instabilities.
%
% that
%
% we can take confidence in the topological treatment for a Kelvin-Helmholtz
% instability.
The interesting aspect of these topological protocols is that the
numerical method comparisons are based on physical differences rather than on
unreliable, low-level, pointwise measures.
% quantitative.
The direction we wish to take now, for our future work, is
the extension of these protocols to 3D datasets of external hypersonics aerodynamics.
% However, in order to isolate one phenomenon on 3D data, a
% threshold study will have to be implemented. It will be necessary to decide
% which phenomenon we want to observe, such as boundary layer turbulence or
% vortices generated at the back of the hypersonic vehicle.
Another direction we
want to investigate is the evaluation of other tools used in the protocol such
as new topological distances \cite{pont_vis21} or clustering methods. Finally, this experimental
study allows us, with confidence, to consider applying these protocols to other
hydrodynamic turbulent flows studied in our institution in the domain of
hypersonic vehicle design.


% \section{Acknowledgments}
%future work:
%benchmark for other topological comparison methods.
