\section{mesh}

\subsection{Généralité maillage}

Il existe différentes catégories de maillages. Ces différents types de maillages peuvent être associables et ils ont chacun des avantages et des inconvénients. Dans cette partie on détaillera les différents maillages en exposant leurs caractéristiques.\\

Les maillages cartésiens sont des maillages orthogonaux. Ils sont construits à partir de lignes à x, y constants en 2D et à x,y,z constants en 3D. Il seront alors composés de quadrangles en 2D et d'hexaèdres en 3D. Ce type de maillage à l'avantage d'être simple à construire et il permet l'utilisation de schémas numériques à ordre élevé lors de calculs en différence finie. L'inconvénient majeur de ce maillage est qu'il n'est pas adaptable pour des géométries courbés.\\
On retrouve ce type de maillage lors de calculs pour les "frontières immergées" où l'on ne prend pas en compte l'objet pour mailler le domaine de calcul.\\

Le maillage structuré en 2D (i,j) ou 3D (i,j,k) est un maillage cartésien qui a été déformé. Comme pour le maillage cartésien il sera alors composé de quadrangles ou d'hexaèdres selon sa dimension. On peut identifier facilement les cellules de ce maillage à l'aide des indices des nœuds qui le compose. En 2D les indices seront des doublets et en 3D se sera des triplets.
Étant similaire au maillage cartésien, il a les mêmes inconvénients. Il est incapable de représenter des géométries complexes comme des chambres de combustion comprenant des trous ou des inclusions.\\
% la difficulté à donner un caractère local à l’adaptation qui se propage dans toutes les directions principales du maillage, comme le montre la figure 1.1.\\
Il est tout de même possible d'adapter un maillage structuré sur des géométries complexes en utilisant la méthode de maillage multibloc. En décomposant le domaine de calcul en plusieurs sous-domaines, appelés blocs. Chaque sous domaine est maillé indépendamment des autres. Cela permet d'adapter le maillage à la géométrie de l'objet.\\

Le dernier type de maillage présenté est le maillage non-structuré. Pour des géométries extrêmement complexes on a recours à ce genre de maillage. La génération d'un maillage pour s'adapter à une géométrie est beaucoup plus simple en non-structuré qu’en structuré. Les éléments qui le composent sont en général des triangles en 2D et des tétraèdres en 3D. Ce maillage est généré arbitrairement et sans contraintes pour la disposition des mailles. Il permet de conserver une bonne qualité des éléments pour les géométries complexes.\\
En revanche il est très couteux en nombres d'éléments et il peut engendrer des erreurs numériques plus importantes qu’un maillage structuré.\\
Il existe un dernier type de maillage qui est l’hybride. Il est la combinaison d'éléments triangulaires et de quadrangles en 2D, les mailles seront des tétraèdres et des hexaèdres en 3D.\\
Ce maillage a les avantages des maillages structurés et non-structurés en réduisant les erreurs numériques. Il est difficile à générer notamment au niveaux des liaisons entre les deux types de maillages.


%Pour mailler certaines géométries très complexes, le recours aux maillages non-structurés est une solution. La génération d’un maillage initial s’adaptant à une géométrie complexe est beaucoup plus simple qu’en structuré. De plus, l’utilisation de maillages non-structurés permet de limiter le raffinement à une zone locale dans le cas général comme expliqué dans le rapport technique de Dervieux et Désidéri [27]. Cependant, l’inconvénient par rapport aux maillages structurés est le coût du stockage de la topologie du maillage. Par exemple, pour une méthode de type « volumes finis » avec les valeurs définies aux centres des cellules, il est en effet nécessaire de stocker une table de connectivité donnant les indices des cellules gauche et droite pour chacune des interfaces du maillage et des tableaux donnant les indices des nœuds qui composent chaque interface et chaque cellule. Ces maillages sont généralement composés de triangles en 2D et de tétraèdres, prismes, pyramides et hexaèdres en 3D. La figure 1.2, extraite de l’article de Dobrzynski et Frey [32], présente un maillage adapté à la prise en compte d’un choc plan dans le domaine de calcul. On y voit bien que le raffinement est localisé au niveau du choc et ne se propage pas au reste du domaine.

\subsection{body fitted}
Le second logiciel est GMSH développé par GMESH de Christophe Geuzaine et Jean-François Remacle de l'université de Louvain. Il a été choisi parce qu'il est possible d'écrire des scripts en Python. De plus, GMSH est un logiciel libre et évolutif. Il sera possible de demander au développeur d'inclure de nouvelles fonctionnalités pour le CEA/CESTA.

Le travail effectué avec GMSH s'est fait sur un environnement linux. Dans un premier temps, il a fallu installer une machine virtuelle pour obtenir linux sur ma machine. Travailler avec ce nouveau bureau m'as permis de me familiariser avec ce système d'exploitation que j'ai utilisé par la suite au CEA.

Le travail effectué avec GMSH s'est fait entièrement sur la machine virtuelle. Ce logiciel de maillage a des avantages et des inconvénients. GMSH est performant pour réaliser des maillages non structurés. Il possède également des capacités en maillage structuré que nous souhaitons évaluer dans le cadre de cette étude.\\
De plus, ce logiciel est utilisable de différentes façons: on peut soit réaliser ses maillages en utilisant l'interface interactive, soit en écrivant un script via des éditeurs de texte ou bien en codant le maillage directement. Chaque méthode est utilisée pour réaliser les différents cas. La version que j'ai utilisée est GMSH 4.5.6.

La méthode interactive permet de produire les géométries et les maillages facilement. On commence par choisir quel type de ressource utiliser pour la création de la géométrie. Nous avons adopté le modeleur basé sur "OpenCascade".\\
La création d'objet en mode interactif se fait en sélectionnant les différentes options et en rentrant les paramètres directement sur l'interface du logiciel comme pour les coordonnées de points par exemple. Néanmoins avec cette méthode nous ne pouvons pas paramétrer les géométries ni le maillage.

La seconde utilisation se fait avec un script, au sein d'un éditeur. Cette méthode est intéressante car elle permet de définir des paramètres comme les rayons des cercles ou la longueur d'un rectangle. De plus, en écrivant son fichier texte, à la moindre erreur nous pouvons la rectifier rapidement.\\
Cette méthode reste intéressante car le langage de description est utilisé pour d'autres logiciels notamment "Openfoam" pour la mécanique des fluides.

La dernière méthode est de développer un code en utilisant l'API GMSH pour réaliser son maillage et sa géométrie. Plusieurs langages sont disponibles comme le C ou le C++, mais nous avons utilisé la langage Python.\\
Pour développer mes codes j'utilise un éditeur, "VScode". Utiliser un éditeur permet d'avoir plus de clarté que lorsqu'on travaille sur la console.\\
En utilisant le langage Python, on peut paramétrer son maillage et la géométrie facilement, il suffit de renseigner les différents arguments. Contrairement à l'utilisation de script nous pouvons utiliser les librairies et les fonctionnalités que nous offre le code Python. Il y a un le fichier python où toutes les commandes sont documentées pour générer les modèles que l'on souhaite, ce qui apporte une aide importante.

\subsection{Immersed boundary method}
